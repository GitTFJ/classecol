\documentclass[]{article}
\usepackage{lmodern}
\usepackage{amssymb,amsmath}
\usepackage{ifxetex,ifluatex}
\usepackage{fixltx2e} % provides \textsubscript
\ifnum 0\ifxetex 1\fi\ifluatex 1\fi=0 % if pdftex
  \usepackage[T1]{fontenc}
  \usepackage[utf8]{inputenc}
\else % if luatex or xelatex
  \ifxetex
    \usepackage{mathspec}
  \else
    \usepackage{fontspec}
  \fi
  \defaultfontfeatures{Ligatures=TeX,Scale=MatchLowercase}
\fi
% use upquote if available, for straight quotes in verbatim environments
\IfFileExists{upquote.sty}{\usepackage{upquote}}{}
% use microtype if available
\IfFileExists{microtype.sty}{%
\usepackage{microtype}
\UseMicrotypeSet[protrusion]{basicmath} % disable protrusion for tt fonts
}{}
\usepackage[margin=1in]{geometry}
\usepackage{hyperref}
\hypersetup{unicode=true,
            pdftitle={classecol: vignette},
            pdfauthor={Thomas Frederick Johnson GitTFJ},
            pdfborder={0 0 0},
            breaklinks=true}
\urlstyle{same}  % don't use monospace font for urls
\usepackage{graphicx,grffile}
\makeatletter
\def\maxwidth{\ifdim\Gin@nat@width>\linewidth\linewidth\else\Gin@nat@width\fi}
\def\maxheight{\ifdim\Gin@nat@height>\textheight\textheight\else\Gin@nat@height\fi}
\makeatother
% Scale images if necessary, so that they will not overflow the page
% margins by default, and it is still possible to overwrite the defaults
% using explicit options in \includegraphics[width, height, ...]{}
\setkeys{Gin}{width=\maxwidth,height=\maxheight,keepaspectratio}
\IfFileExists{parskip.sty}{%
\usepackage{parskip}
}{% else
\setlength{\parindent}{0pt}
\setlength{\parskip}{6pt plus 2pt minus 1pt}
}
\setlength{\emergencystretch}{3em}  % prevent overfull lines
\providecommand{\tightlist}{%
  \setlength{\itemsep}{0pt}\setlength{\parskip}{0pt}}
\setcounter{secnumdepth}{0}
% Redefines (sub)paragraphs to behave more like sections
\ifx\paragraph\undefined\else
\let\oldparagraph\paragraph
\renewcommand{\paragraph}[1]{\oldparagraph{#1}\mbox{}}
\fi
\ifx\subparagraph\undefined\else
\let\oldsubparagraph\subparagraph
\renewcommand{\subparagraph}[1]{\oldsubparagraph{#1}\mbox{}}
\fi

%%% Use protect on footnotes to avoid problems with footnotes in titles
\let\rmarkdownfootnote\footnote%
\def\footnote{\protect\rmarkdownfootnote}

%%% Change title format to be more compact
\usepackage{titling}

% Create subtitle command for use in maketitle
\providecommand{\subtitle}[1]{
  \posttitle{
    \begin{center}\large#1\end{center}
    }
}

\setlength{\droptitle}{-2em}

  \title{classecol: vignette}
    \pretitle{\vspace{\droptitle}\centering\huge}
  \posttitle{\par}
    \author{Thomas Frederick Johnson GitTFJ}
    \preauthor{\centering\large\emph}
  \postauthor{\par}
      \predate{\centering\large\emph}
  \postdate{\par}
    \date{6 July 2020}


\begin{document}
\maketitle

\section{classecol}\label{classecol}

classecol is a package to perform text classifications of twitter data
(but may be useful for any social media text). It provides a series of
functions to clean data, produces a sentiment matrix of some of the most
popular methods, and has three models to classify text: biographical -
to assess who the user is, environment - to assess if the tweet is
related to their environment at the users stance on the enviornment,
hunting - to assess if the tweet is related to hunting and determine
their stance on hunting.

\subsection{Intstructions to prepare
package}\label{intstructions-to-prepare-package}

This package is runs through R but is reliant on a python back-end
(which dramitcally improves the speed of the classification). So before
running any code, you will need to install a version of python - we
recommend Python 3.6.9 which is what the package has been tested on;
available
\href{https://www.python.org/downloads/release/python-369/}{here}. You
will also need to install a selection of packages on python: numpy, os,
pandas, re, nltk, bs4, string, joblib, pickle, sklearn, keras,
tensorflow, and time. Packages can be installed following these
\href{https://packaging.python.org/tutorials/installing-packages/}{instructions}.
If at anypoint the error `Module not found' appears, install this
module/package following these ihe above instructions.

classecol is only available as a github repository so needs to be
installed through github

\begin{verbatim}
library(devtools)
install_github("GitTFJ/classecol")
library(classecol)
library(reticulate)
\end{verbatim}

classecol is not self-contained is alos reliant on a prtner repository
which stores the python models and code. This section of code will
download the accompanying repository and save it into a specified
location

\begin{verbatim}
#dir.create("models")
#download_models("models")
\end{verbatim}

In order to model the text file, we need to link R to the python
program. `reticulate' offers a function to perform this, but we found it
performed incosistently, so require manuallly specifying pythons
absolute filepath location. the file to search for is `python.exe'.

You will also need to specify the location you have downloaded the
models to and then send this location to python with the function
`r\_to\_py'

\begin{verbatim}
reticulate::use_python("C:/Users/mn826766/Anaconda3/python.exe")
setwd("C:/Users/mn826766/OneDrive - University of Reading/PhDResearch/UnderstandingDeclinesInLargeCarnivores/Chapters/Twitter/ClassifyTweets/Manuscript")
direc = paste(getwd(),"/", "models/classecol-models-master/", sep = "")
model_directory = reticulate::r_to_py(direc)
\end{verbatim}

\subsection{Biographical classifier}\label{biographical-classifier}

At this point we are ready to prepare and classify the data. The
biographical classifier `bio\_class()' works best with twitter data in
its dirty form, so none of the text should be cleaned. However, it is
neccasary to join the twitter name and description into one column named
`text' split with a space. if the column is named anything but `text'
with a dataframe named `data' the download will fail.

\begin{verbatim}
df = data.frame(
  name = c(
    "Boris Johnson #StayAlert",
    "Manuela Gonzalez",
    "University of Reading"),
  description = c(
  "Prime Minister of the United Kingdom and @Conservatives leader. Member of Parliament for Uxbridge and South Ruislip. #StayAlert", 
  "Quantitative ecologist interested in conservation & population dynamics. Lecturer at University of Reading, UK. She/her", 
  "Campus life and study at the University of Reading, UK. For news and comment follow @UniRdg_News"))
df$text = paste(df$name, df$description)
data = reticulate::r_to_py(df)
bio_class(
  type = "split",
  directory = direc)
\end{verbatim}

\subsection{Hunting classifier}\label{hunting-classifier}

The hunting classifier `hunt\_class()' works best with twitter data
after a simple clean. if the column is named anything but `text' with a
dataframe named `data' the download will fail.

\begin{verbatim}
df = data.frame(
  text = c(
    "I hate hunting", 
    "Cant wait to go hunting", 
    "Hunting for my car keys"))
df$text = classecol::clean(df$text, level = "simple")
data = reticulate::r_to_py(df)
hunt_class(
  type = "all",
  directory = direc)
\end{verbatim}

\subsection{Environment classifier}\label{environment-classifier}

The environment classifier `env\_class()' works best with twitter data
after a full clean and also requires sentiment analysis on the text. if
the column is named anything but `text' with a dataframe named `data'
the download will fail.

\begin{verbatim}
df = data.frame(
  text = c(
    "I love walking in nature", 
    "I am so sad we losing the rainforest. stop the destruction", 
    "Tiger wins the PGA tour again!"))
df$text = classecol::clean(df$text, level = "full")
sm = as.matrix(cbind(
  valence(df$text), 
  lang_eng(as.character(df$text)), 
  senti_matrix(as.character(contract(df$text)))))
data = reticulate::r_to_py(df)
sent_mat = reticulate::r_to_py(sm)
env_class(
  type = "trim",
  directory = direc)
\end{verbatim}


\end{document}
